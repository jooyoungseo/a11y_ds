% Options for packages loaded elsewhere
\PassOptionsToPackage{unicode}{hyperref}
\PassOptionsToPackage{hyphens}{url}
\PassOptionsToPackage{dvipsnames,svgnames,x11names}{xcolor}
%
\documentclass[
  letterpaper,
  DIV=11,
  numbers=noendperiod]{scrartcl}

\usepackage{amsmath,amssymb}
\usepackage{iftex}
\ifPDFTeX
  \usepackage[T1]{fontenc}
  \usepackage[utf8]{inputenc}
  \usepackage{textcomp} % provide euro and other symbols
\else % if luatex or xetex
  \usepackage{unicode-math}
  \defaultfontfeatures{Scale=MatchLowercase}
  \defaultfontfeatures[\rmfamily]{Ligatures=TeX,Scale=1}
\fi
\usepackage{lmodern}
\ifPDFTeX\else  
    % xetex/luatex font selection
\fi
% Use upquote if available, for straight quotes in verbatim environments
\IfFileExists{upquote.sty}{\usepackage{upquote}}{}
\IfFileExists{microtype.sty}{% use microtype if available
  \usepackage[]{microtype}
  \UseMicrotypeSet[protrusion]{basicmath} % disable protrusion for tt fonts
}{}
\makeatletter
\@ifundefined{KOMAClassName}{% if non-KOMA class
  \IfFileExists{parskip.sty}{%
    \usepackage{parskip}
  }{% else
    \setlength{\parindent}{0pt}
    \setlength{\parskip}{6pt plus 2pt minus 1pt}}
}{% if KOMA class
  \KOMAoptions{parskip=half}}
\makeatother
\usepackage{xcolor}
\setlength{\emergencystretch}{3em} % prevent overfull lines
\setcounter{secnumdepth}{-\maxdimen} % remove section numbering
% Make \paragraph and \subparagraph free-standing
\ifx\paragraph\undefined\else
  \let\oldparagraph\paragraph
  \renewcommand{\paragraph}[1]{\oldparagraph{#1}\mbox{}}
\fi
\ifx\subparagraph\undefined\else
  \let\oldsubparagraph\subparagraph
  \renewcommand{\subparagraph}[1]{\oldsubparagraph{#1}\mbox{}}
\fi

\usepackage{color}
\usepackage{fancyvrb}
\newcommand{\VerbBar}{|}
\newcommand{\VERB}{\Verb[commandchars=\\\{\}]}
\DefineVerbatimEnvironment{Highlighting}{Verbatim}{commandchars=\\\{\}}
% Add ',fontsize=\small' for more characters per line
\usepackage{framed}
\definecolor{shadecolor}{RGB}{241,243,245}
\newenvironment{Shaded}{\begin{snugshade}}{\end{snugshade}}
\newcommand{\AlertTok}[1]{\textcolor[rgb]{0.68,0.00,0.00}{#1}}
\newcommand{\AnnotationTok}[1]{\textcolor[rgb]{0.37,0.37,0.37}{#1}}
\newcommand{\AttributeTok}[1]{\textcolor[rgb]{0.40,0.45,0.13}{#1}}
\newcommand{\BaseNTok}[1]{\textcolor[rgb]{0.68,0.00,0.00}{#1}}
\newcommand{\BuiltInTok}[1]{\textcolor[rgb]{0.00,0.23,0.31}{#1}}
\newcommand{\CharTok}[1]{\textcolor[rgb]{0.13,0.47,0.30}{#1}}
\newcommand{\CommentTok}[1]{\textcolor[rgb]{0.37,0.37,0.37}{#1}}
\newcommand{\CommentVarTok}[1]{\textcolor[rgb]{0.37,0.37,0.37}{\textit{#1}}}
\newcommand{\ConstantTok}[1]{\textcolor[rgb]{0.56,0.35,0.01}{#1}}
\newcommand{\ControlFlowTok}[1]{\textcolor[rgb]{0.00,0.23,0.31}{#1}}
\newcommand{\DataTypeTok}[1]{\textcolor[rgb]{0.68,0.00,0.00}{#1}}
\newcommand{\DecValTok}[1]{\textcolor[rgb]{0.68,0.00,0.00}{#1}}
\newcommand{\DocumentationTok}[1]{\textcolor[rgb]{0.37,0.37,0.37}{\textit{#1}}}
\newcommand{\ErrorTok}[1]{\textcolor[rgb]{0.68,0.00,0.00}{#1}}
\newcommand{\ExtensionTok}[1]{\textcolor[rgb]{0.00,0.23,0.31}{#1}}
\newcommand{\FloatTok}[1]{\textcolor[rgb]{0.68,0.00,0.00}{#1}}
\newcommand{\FunctionTok}[1]{\textcolor[rgb]{0.28,0.35,0.67}{#1}}
\newcommand{\ImportTok}[1]{\textcolor[rgb]{0.00,0.46,0.62}{#1}}
\newcommand{\InformationTok}[1]{\textcolor[rgb]{0.37,0.37,0.37}{#1}}
\newcommand{\KeywordTok}[1]{\textcolor[rgb]{0.00,0.23,0.31}{#1}}
\newcommand{\NormalTok}[1]{\textcolor[rgb]{0.00,0.23,0.31}{#1}}
\newcommand{\OperatorTok}[1]{\textcolor[rgb]{0.37,0.37,0.37}{#1}}
\newcommand{\OtherTok}[1]{\textcolor[rgb]{0.00,0.23,0.31}{#1}}
\newcommand{\PreprocessorTok}[1]{\textcolor[rgb]{0.68,0.00,0.00}{#1}}
\newcommand{\RegionMarkerTok}[1]{\textcolor[rgb]{0.00,0.23,0.31}{#1}}
\newcommand{\SpecialCharTok}[1]{\textcolor[rgb]{0.37,0.37,0.37}{#1}}
\newcommand{\SpecialStringTok}[1]{\textcolor[rgb]{0.13,0.47,0.30}{#1}}
\newcommand{\StringTok}[1]{\textcolor[rgb]{0.13,0.47,0.30}{#1}}
\newcommand{\VariableTok}[1]{\textcolor[rgb]{0.07,0.07,0.07}{#1}}
\newcommand{\VerbatimStringTok}[1]{\textcolor[rgb]{0.13,0.47,0.30}{#1}}
\newcommand{\WarningTok}[1]{\textcolor[rgb]{0.37,0.37,0.37}{\textit{#1}}}

\providecommand{\tightlist}{%
  \setlength{\itemsep}{0pt}\setlength{\parskip}{0pt}}\usepackage{longtable,booktabs,array}
\usepackage{calc} % for calculating minipage widths
% Correct order of tables after \paragraph or \subparagraph
\usepackage{etoolbox}
\makeatletter
\patchcmd\longtable{\par}{\if@noskipsec\mbox{}\fi\par}{}{}
\makeatother
% Allow footnotes in longtable head/foot
\IfFileExists{footnotehyper.sty}{\usepackage{footnotehyper}}{\usepackage{footnote}}
\makesavenoteenv{longtable}
\usepackage{graphicx}
\makeatletter
\def\maxwidth{\ifdim\Gin@nat@width>\linewidth\linewidth\else\Gin@nat@width\fi}
\def\maxheight{\ifdim\Gin@nat@height>\textheight\textheight\else\Gin@nat@height\fi}
\makeatother
% Scale images if necessary, so that they will not overflow the page
% margins by default, and it is still possible to overwrite the defaults
% using explicit options in \includegraphics[width, height, ...]{}
\setkeys{Gin}{width=\maxwidth,height=\maxheight,keepaspectratio}
% Set default figure placement to htbp
\makeatletter
\def\fps@figure{htbp}
\makeatother

\KOMAoption{captions}{tableheading}
\makeatletter
\makeatother
\makeatletter
\makeatother
\makeatletter
\@ifpackageloaded{caption}{}{\usepackage{caption}}
\AtBeginDocument{%
\ifdefined\contentsname
  \renewcommand*\contentsname{Table of contents}
\else
  \newcommand\contentsname{Table of contents}
\fi
\ifdefined\listfigurename
  \renewcommand*\listfigurename{List of Figures}
\else
  \newcommand\listfigurename{List of Figures}
\fi
\ifdefined\listtablename
  \renewcommand*\listtablename{List of Tables}
\else
  \newcommand\listtablename{List of Tables}
\fi
\ifdefined\figurename
  \renewcommand*\figurename{Figure}
\else
  \newcommand\figurename{Figure}
\fi
\ifdefined\tablename
  \renewcommand*\tablename{Table}
\else
  \newcommand\tablename{Table}
\fi
}
\@ifpackageloaded{float}{}{\usepackage{float}}
\floatstyle{ruled}
\@ifundefined{c@chapter}{\newfloat{codelisting}{h}{lop}}{\newfloat{codelisting}{h}{lop}[chapter]}
\floatname{codelisting}{Listing}
\newcommand*\listoflistings{\listof{codelisting}{List of Listings}}
\makeatother
\makeatletter
\@ifpackageloaded{caption}{}{\usepackage{caption}}
\@ifpackageloaded{subcaption}{}{\usepackage{subcaption}}
\makeatother
\makeatletter
\makeatother
\ifLuaTeX
  \usepackage{selnolig}  % disable illegal ligatures
\fi
\IfFileExists{bookmark.sty}{\usepackage{bookmark}}{\usepackage{hyperref}}
\IfFileExists{xurl.sty}{\usepackage{xurl}}{} % add URL line breaks if available
\urlstyle{same} % disable monospaced font for URLs
\hypersetup{
  pdftitle={Session 1: Installation Guide for R and Visual Studio Code for Screen Reader Users},
  pdfauthor={JooYoung Seo},
  colorlinks=true,
  linkcolor={blue},
  filecolor={Maroon},
  citecolor={Blue},
  urlcolor={Blue},
  pdfcreator={LaTeX via pandoc}}

\title{Session 1: Installation Guide for R and Visual Studio Code for
Screen Reader Users}
\author{JooYoung Seo}
\date{2023-08-02}

\begin{document}
\maketitle
\hypertarget{sec-install-r-vscode}{%
\section{Install R, Visual Studio Code, and Related
Dependencies}\label{sec-install-r-vscode}}

this section describes how to install R, Visual Studio Code, and other
related dependencies. The following instructions are tested on Windows
10 and Mac OS.

\hypertarget{windows}{%
\subsection{Windows}\label{windows}}

\begin{enumerate}
\def\labelenumi{\arabic{enumi}.}
\item
  Press \texttt{Windows+R} and type ``cmd'' without the quotes. After
  typing, don't hit \texttt{Enter}. Instead, press
  \texttt{Ctrl+Shift+Enter} to run the command prompt as administrator
  privilege. If you are prompted for an administrator password or
  confirmation, type the password or provide confirmation.
\item
  Copy and paste the following command into the command prompt and press
  enter. This will install chocolatey, a package manager for Windows.
  Note that there is copy button below the command. You can click the
  copy button to copy the command. To paste the command into the command
  prompt, you can press Ctrl+V.
\end{enumerate}

\begin{Shaded}
\begin{Highlighting}[]
\ExtensionTok{@}\StringTok{"\%SystemRoot\%\textbackslash{}System32\textbackslash{}WindowsPowerShell\textbackslash{}v1.0\textbackslash{}powershell.exe"} \AttributeTok{{-}NoProfile} \AttributeTok{{-}InputFormat}\NormalTok{ None }\AttributeTok{{-}ExecutionPolicy}\NormalTok{ Bypass }\AttributeTok{{-}Command} \StringTok{"[System.Net.ServicePointManager]::SecurityProtocol = 3072; iex ((New{-}Object System.Net.WebClient).DownloadString(\textquotesingle{}https://community.chocolatey.org/install.ps1\textquotesingle{}))"} \KeywordTok{\&\&} \ExtensionTok{SET} \StringTok{"PATH=\%PATH\%;\%ALLUSERSPROFILE\%\textbackslash{}chocolatey\textbackslash{}bin"}
\end{Highlighting}
\end{Shaded}

\begin{enumerate}
\def\labelenumi{\arabic{enumi}.}
\setcounter{enumi}{2}
\tightlist
\item
  Copy and paste the following command into the opened command prompt
  and press enter. This will install R, VSCode and other related
  dependencies. Grab a cup of coffee and wait for the installation to
  finish. It may take a while. You will hear a ``do mi sol'' beep sound
  when the installation is completed. If it fails to install, you will
  hear a ``do do do'' beep sound. If you hear the ``do do do'' beep
  sound, you can try to run the command again.
\end{enumerate}

\begin{Shaded}
\begin{Highlighting}[]
\ExtensionTok{powershell} \AttributeTok{{-}Command} \StringTok{"\& \{ }\VariableTok{$sapi}\StringTok{ = New{-}Object {-}ComObject SAPI.SpVoice; }\VariableTok{$sapi}\StringTok{.Rate = 10; }\VariableTok{$packages}\StringTok{ = @(\textquotesingle{}tinytex\textquotesingle{}, \textquotesingle{}pandoc\textquotesingle{}, \textquotesingle{}quarto\textquotesingle{}, \textquotesingle{}rtools\textquotesingle{}, \textquotesingle{}r.project\textquotesingle{}, \textquotesingle{}vscode\textquotesingle{}); }\VariableTok{$executables}\StringTok{ = @(\textquotesingle{}tlmgr\textquotesingle{}, \textquotesingle{}pandoc\textquotesingle{}, \textquotesingle{}quarto\textquotesingle{}, \textquotesingle{}\textquotesingle{}, \textquotesingle{}R\textquotesingle{}, \textquotesingle{}code\textquotesingle{}); }\VariableTok{$installPackages}\StringTok{ = @(); }\VariableTok{$installedPackages}\StringTok{ = @(); foreach (}\VariableTok{$package}\StringTok{ in }\VariableTok{$packages}\StringTok{) \{ }\VariableTok{$index}\StringTok{ = }\VariableTok{$packages}\StringTok{.IndexOf(}\VariableTok{$package}\StringTok{); if (}\VariableTok{$package}\StringTok{ {-}eq \textquotesingle{}rtools\textquotesingle{}) \{ if ([Environment]::GetEnvironmentVariable(\textquotesingle{}RTOOLS43\_HOME\textquotesingle{}, \textquotesingle{}User\textquotesingle{}) {-}eq }\VariableTok{$null}\StringTok{ {-}and [Environment]::GetEnvironmentVariable(\textquotesingle{}RTOOLS43\_HOME\textquotesingle{}, \textquotesingle{}Machine\textquotesingle{}) {-}eq }\VariableTok{$null}\StringTok{) \{ }\VariableTok{$installPackages}\StringTok{ += }\VariableTok{$package}\StringTok{ \} else \{ }\VariableTok{$installedPackages}\StringTok{ += }\VariableTok{$package}\StringTok{ \} \} else \{ if ((Get{-}Command }\VariableTok{$executables}\StringTok{[}\VariableTok{$index}\StringTok{] {-}ErrorAction SilentlyContinue) {-}eq }\VariableTok{$null}\StringTok{) \{ }\VariableTok{$installPackages}\StringTok{ += }\VariableTok{$package}\StringTok{ \} else \{ }\VariableTok{$installedPackages}\StringTok{ += }\VariableTok{$package}\StringTok{ \} \} \}; if (}\VariableTok{$installedPackages}\StringTok{.Count {-}gt 0) \{ [console]::beep(261, 100); Start{-}Sleep {-}Milliseconds 100; [console]::beep(329, 100); Start{-}Sleep {-}Milliseconds 100; [console]::beep(392, 100); }\VariableTok{$sapi}\StringTok{.Speak(\textquotesingle{}The following packages are already installed: \textquotesingle{} + (}\VariableTok{$installedPackages}\StringTok{ {-}join \textquotesingle{}, \textquotesingle{})) \}; if (}\VariableTok{$installPackages}\StringTok{.Count {-}gt 0) \{ [console]::beep(261, 100); Start{-}Sleep {-}Milliseconds 100; [console]::beep(329, 100); Start{-}Sleep {-}Milliseconds 100; [console]::beep(392, 100); }\VariableTok{$sapi}\StringTok{.Speak(\textquotesingle{}The following packages will be installed: \textquotesingle{} + (}\VariableTok{$installPackages}\StringTok{ {-}join \textquotesingle{}, \textquotesingle{})); choco install }\VariableTok{$installPackages}\StringTok{ {-}y \}; if(}\VariableTok{$installedPackages}\StringTok{.Count {-}eq 0 {-}and }\VariableTok{$installPackages}\StringTok{.Count {-}eq 0) \{[console]::beep(261, 100); Start{-}Sleep {-}Milliseconds 100; [console]::beep(329, 100); Start{-}Sleep {-}Milliseconds 100; [console]::beep(392, 100); }\VariableTok{$sapi}\StringTok{.Speak(\textquotesingle{}All packages are already installed.\textquotesingle{})\} \}"}
\end{Highlighting}
\end{Shaded}

\begin{enumerate}
\def\labelenumi{\arabic{enumi}.}
\setcounter{enumi}{3}
\tightlist
\item
  After installation, type the following in the opened command prompt
  and hit enter key. This will add R to your path environment variable.
  You will hear a ``do mi sol'' beep sound when the installation is
  completed. If it fails to install, you will hear a ``do do do'' beep
  sound. If you hear the ``do do do'' beep sound, you can try to run the
  command again.
\end{enumerate}

\begin{Shaded}
\begin{Highlighting}[]
\ExtensionTok{powershell} \AttributeTok{{-}Command} \StringTok{"\& \{ }\VariableTok{$sapi}\StringTok{ = New{-}Object {-}ComObject SAPI.SpVoice; }\VariableTok{$sapi}\StringTok{.Rate = 10; try \{ }\VariableTok{$RInstalls}\StringTok{ = Get{-}ChildItem \textquotesingle{}C:\textbackslash{}Program Files\textbackslash{}R\textquotesingle{} {-}Directory; }\VariableTok{$LatestR}\StringTok{ = }\VariableTok{$RInstalls}\StringTok{ | Sort{-}Object LastWriteTime {-}Descending | Select{-}Object {-}First 1; }\VariableTok{$RBinPath}\StringTok{ = Join{-}Path {-}Path }\VariableTok{$LatestR}\StringTok{.FullName {-}ChildPath \textquotesingle{}bin\textbackslash{}x64\textquotesingle{}; }\VariableTok{$RSetRegPath}\StringTok{ = Join{-}Path {-}Path }\VariableTok{$LatestR}\StringTok{.FullName {-}ChildPath \textquotesingle{}bin\textbackslash{}x64\textbackslash{}RSetReg.exe\textquotesingle{}; \& }\VariableTok{$RSetRegPath}\StringTok{; }\VariableTok{$UserPath}\StringTok{ = [Environment]::GetEnvironmentVariable(\textquotesingle{}Path\textquotesingle{}, [System.EnvironmentVariableTarget]::User); }\VariableTok{$SystemPath}\StringTok{ = [Environment]::GetEnvironmentVariable(\textquotesingle{}Path\textquotesingle{}, [System.EnvironmentVariableTarget]::Machine); if ((}\VariableTok{$UserPath}\StringTok{ {-}split \textquotesingle{};\textquotesingle{} {-}notcontains }\VariableTok{$RBinPath}\StringTok{) {-}and (}\VariableTok{$SystemPath}\StringTok{ {-}split \textquotesingle{};\textquotesingle{} {-}notcontains }\VariableTok{$RBinPath}\StringTok{)) \{ }\VariableTok{$NewPath}\StringTok{ = }\VariableTok{$UserPath}\StringTok{ + \textquotesingle{};\textquotesingle{} + }\VariableTok{$RBinPath}\StringTok{; [Environment]::SetEnvironmentVariable(\textquotesingle{}Path\textquotesingle{}, }\VariableTok{$NewPath}\StringTok{, [System.EnvironmentVariableTarget]::User) \}; [console]::beep(261, 100); Start{-}Sleep {-}Milliseconds 100; [console]::beep(329, 100); Start{-}Sleep {-}Milliseconds 100; [console]::beep(392, 100); }\VariableTok{$sapi}\StringTok{.Speak(\textquotesingle{}Process Completed Successfully\textquotesingle{}) \} catch \{ [console]::beep(261, 100); Start{-}Sleep {-}Milliseconds 100; [console]::beep(261, 100); Start{-}Sleep {-}Milliseconds 100; [console]::beep(261, 100); }\VariableTok{$sapi}\StringTok{.Speak(\textquotesingle{}An Error Occurred During Execution\textquotesingle{}) \}\}"}
\end{Highlighting}
\end{Shaded}

\begin{enumerate}
\def\labelenumi{\arabic{enumi}.}
\setcounter{enumi}{4}
\item
  Close the command prompt by pressing \texttt{Alt+F4} key.
\item
  Press \texttt{Windows+R} and type ``cmd'' without the quotes. After
  typing, don't hit \texttt{Enter}. Instead, press
  \texttt{Ctrl+Shift+Enter} to run the command prompt as administrator
  privilege. If you are prompted for an administrator password or
  confirmation, type the password or provide confirmation.
\item
  In the opened command prompt, copy and paste the following command and
  press enter. This will install R packages that are required for
  VSCode.
\end{enumerate}

\begin{Shaded}
\begin{Highlighting}[]
\SpecialCharTok{@}\NormalTok{refreshenv }\SpecialCharTok{\&\&}\NormalTok{ Rscript }\SpecialCharTok{{-}}\NormalTok{e }\StringTok{"packages \textless{}{-} c(\textquotesingle{}languageserver\textquotesingle{}, \textquotesingle{}lintr\textquotesingle{}, \textquotesingle{}httpgd\textquotesingle{}, \textquotesingle{}DT\textquotesingle{}, \textquotesingle{}beepr\textquotesingle{}, \textquotesingle{}devtools\textquotesingle{}); installed\_packages \textless{}{-} rownames(installed.packages()); packages\_to\_install \textless{}{-} setdiff(packages, installed\_packages); if(length(packages\_to\_install) \textgreater{} 0) \{ install.packages(packages\_to\_install, repos = \textquotesingle{}https://cloud.r{-}project.org/\textquotesingle{}) \}; if(!\textquotesingle{}vscDebugger\textquotesingle{} \%in\% installed\_packages) \{ if(!require(remotes)) \{ install.packages(\textquotesingle{}remotes\textquotesingle{}, repos = \textquotesingle{}https://cloud.r{-}project.org/\textquotesingle{}) \}; remotes::install\_github(\textquotesingle{}ManuelHentschel/vscDebugger\textquotesingle{}) \}"}
\end{Highlighting}
\end{Shaded}

\begin{enumerate}
\def\labelenumi{\arabic{enumi}.}
\setcounter{enumi}{7}
\tightlist
\item
  The following is optional. However, I recommend you to install the
  latest version of PowerShell to benefit from VSCode shell integration
  feature. You can copy and paste the following command into the opened
  command prompt and press enter. Note that the default PowerShell
  version on Windows 10 is 5.1. The following command will install
  PowerShell 7 or above.
\end{enumerate}

\begin{Shaded}
\begin{Highlighting}[]
\ExtensionTok{winget}\NormalTok{ install Microsoft.PowerShell}
\end{Highlighting}
\end{Shaded}

\begin{enumerate}
\def\labelenumi{\arabic{enumi}.}
\setcounter{enumi}{8}
\tightlist
\item
  Go to Section~\ref{sec-vscode-configuration}, to complete the
  remaining steps.
\end{enumerate}

\hypertarget{mac-os}{%
\subsection{Mac OS}\label{mac-os}}

The following assumes that you have installed Homebrew on your system.
If you haven't installed Homebrew, you can install it by following the
instruction described in the \href{https://brew.sh/}{Homebrew website}.

\begin{enumerate}
\def\labelenumi{\arabic{enumi}.}
\item
  Open terminal by pressing \texttt{Command+Space} and type ``terminal''
  without the quotes. Press enter to open the terminal.
\item
  Copy and paste the following command into the terminal and press
  enter. Note that there is copy button below the command. You can click
  the copy button to copy the command. To paste the command into the
  command prompt, you can press CMD+V. Some commands may require you to
  enter your password. If so, you will be prompted to enter your
  password. You will not see any characters when you type your password.
  Just type your password and press enter.
\end{enumerate}

\begin{Shaded}
\begin{Highlighting}[]
\CommentTok{\#  Install R}
\ExtensionTok{brew}\NormalTok{ install r}

\CommentTok{\# Install Pandoc}
\ExtensionTok{brew}\NormalTok{ install pandoc}

\CommentTok{\# Install Quarto}
\ExtensionTok{brew}\NormalTok{ install quarto}

\CommentTok{\# Install VSCode}
\ExtensionTok{brew}\NormalTok{ install }\AttributeTok{{-}{-}cask}\NormalTok{ visual{-}studio{-}code}

\CommentTok{\# Install xcode command line tools}
\ExtensionTok{xcode{-}select} \AttributeTok{{-}{-}install}

\CommentTok{\# Install TinyTeX}
\ExtensionTok{quarto}\NormalTok{ install tinytex}
\end{Highlighting}
\end{Shaded}

\begin{enumerate}
\def\labelenumi{\arabic{enumi}.}
\setcounter{enumi}{2}
\tightlist
\item
  In the terminal, copy and paste the following command and press enter.
  This will install R packages that are required for VSCode.
\end{enumerate}

\begin{Shaded}
\begin{Highlighting}[]
\ExtensionTok{Rscript} \AttributeTok{{-}e} \StringTok{\textquotesingle{}packages \textless{}{-} c("languageserver", "lintr", "httpgd", "DT", "beepr", "devtools"); installed\_packages \textless{}{-} rownames(installed.packages()); packages\_to\_install \textless{}{-} setdiff(packages, installed\_packages); if(length(packages\_to\_install) \textgreater{} 0) \{ install.packages(packages\_to\_install, repos = "https://cloud.r{-}project.org/") \}; if(!"vscDebugger" \%in\% installed\_packages) \{ if(!require(remotes)) \{ install.packages("remotes", repos = "https://cloud.r{-}project.org/") \}; remotes::install\_github("ManuelHentschel/vscDebugger") \}\textquotesingle{}}
\end{Highlighting}
\end{Shaded}

\begin{enumerate}
\def\labelenumi{\arabic{enumi}.}
\setcounter{enumi}{3}
\tightlist
\item
  Go to Section~\ref{sec-vscode-configuration}, to complete the
  remaining steps.
\end{enumerate}

\hypertarget{sec-vscode-configuration}{%
\section{Visual Studio Code
Configuration}\label{sec-vscode-configuration}}

Once you have installed R and Visual Studio Code following the
instruction described in Section~\ref{sec-install-r-vscode}, you need to
configure VSCode to make it accessible. I have created an accessible
VSCode profile for data science work for your convenience. This profile
contains a set of keyboard shortcuts, settings, and extensions that are
useful for screen reader users when working with R and Python. You can
apply this profile by following the steps below:

\begin{enumerate}
\def\labelenumi{\arabic{enumi}.}
\item
  Open VSCode. You can do this by pressing \texttt{Windows+R} and type
  ``code'' without the quotes and press enter. On Mac, you can press
  \texttt{Command+Space} and type ``code'' without the quotes and press
  enter.
\item
  In VSCode, press \texttt{Ctrl+Shift+P} (on Windows) or
  \texttt{Command+Shift+P} (on Mac) to open the command palette.
\item
  Type ``import profile'' without the quotes and press enter.
\item
  If you hear ``Provide Profile Template URL - Import from Profile
  Template\ldots{}'' copy and paste the following URL into the opened
  input box and press enter.
\end{enumerate}

\begin{Shaded}
\begin{Highlighting}[]
\ExtensionTok{https://insiders.vscode.dev/profile/github/0cbbd65e63c766f7d37586e246d033f6}
\end{Highlighting}
\end{Shaded}

\begin{enumerate}
\def\labelenumi{\arabic{enumi}.}
\setcounter{enumi}{4}
\item
  Press \texttt{Ctrl+0} (on Windows) or \texttt{Command+0} (on Mac) to
  open the sidebar. You will hear ``data\_science\_accessible Tree
  View.''
\item
  Press \texttt{Ctrl+LeftArrow} (on Windows) or
  \texttt{Command+LeftArrow} (on Mac) to collapse all the opened tree
  views in the sidebar.
\item
  Keep all the checkboxes checked and press \texttt{Tab} key multiple
  times until you hear ``Create Profile'' button and press
  \texttt{Enter} key.
\end{enumerate}

To verify whether the profile is applied correctly, you can check the
title bar of VSCode. If you hear ``data science accessible - Visual
Studio Code,'' the profile is applied correctly.

\hypertarget{sec-rprofile-settings}{%
\section{RProfile Settings}\label{sec-rprofile-settings}}

After configuring VSCode accessibility as described in
Section~\ref{sec-vscode-configuration}, you need to add the following
code to your \texttt{Rprofile}. The \texttt{Rprofile} is a script that R
runs at startup. You can use this script to customize R startup behavior
to suit your personal preferences.

I have created an accessible \texttt{Rprofile} for your convenience.
This \texttt{Rprofile} contains a set of options that are useful for
screen reader users. To apply this \texttt{Rprofile}, folow the steps
below:

\begin{enumerate}
\def\labelenumi{\arabic{enumi}.}
\item
  Open VSCode. You can do this by pressing \texttt{Windows+R} and type
  ``code'' without the quotes and press enter. On Mac, you can press
  \texttt{Command+Space} and type ``code'' without the quotes and press
  enter.
\item
  In VSCode, press \texttt{Ctrl+Shift+P} (on Windows) or
  \texttt{Command+Shift+P} (on Mac) to open the command palette.
\item
  Type ``create r terminal'' without the quotes and press enter.
\item
  In the opened R terminal, type the following command and press enter.
  This will open your \texttt{Rprofile} in VSCode.
\end{enumerate}

\begin{Shaded}
\begin{Highlighting}[]
\NormalTok{usethis}\SpecialCharTok{::}\FunctionTok{edit\_r\_profile}\NormalTok{()}
\end{Highlighting}
\end{Shaded}

\begin{enumerate}
\def\labelenumi{\arabic{enumi}.}
\setcounter{enumi}{4}
\item
  In the opened \texttt{Rprofile}, press \texttt{Ctrl+End} (on Windows)
  or \texttt{Command+DownArrow} (on Mac) to move to the end of the file.
\item
  Copy and paste the following code to the end of the file. Please don't
  manually copy the code as it is too long. Instead, use the copy button
  below the code to copy the code. To paste the code into the
  \texttt{Rprofile}, you can press \texttt{Ctrl+V} (on Windows) or
  \texttt{Command+V} (on Mac).
\end{enumerate}

\begin{Shaded}
\begin{Highlighting}[]
\CommentTok{\# Useful option 1}
\DocumentationTok{\#\# Setting text{-}based interaction instead of R GUI dialog box; especially, useful when choosing CRAN mirror server for package installation.}
\DocumentationTok{\#\# Since R GUI dialog box is not accesible, blind R users tend to pre{-}define their CRAN mirror server using \textasciigrave{}chooseCRANmirror(ind = 1)\textasciigrave{}; however, you can now access the CRAN server list interactively using the following option.}
\DocumentationTok{\#\# Furthermore, when updating R packages, users often encounter inaccessible dialog box asking about whether to also update other dependencies, if any. The following option also resolves that issue by allowing users to choose based on text UI.}
\FunctionTok{options}\NormalTok{(}\AttributeTok{menu.graphics =} \ConstantTok{FALSE}\NormalTok{)}

\CommentTok{\# Useful option 2}
\DocumentationTok{\#\# Setting pager for your preferable text editor instead of the R internal one which is inaccessible ; especially useful when calling such R functions employing pager option under the hood as \textasciigrave{}data()\textasciigrave{} function.}
\DocumentationTok{\#\# For example, if you would like to know what dataset are available under "ggplot2" package, you might type in \textasciigrave{}data(package = "ggplot2")\textasciigrave{}; however, the result viewer displayed in the internal pager is not accessible.}
\DocumentationTok{\#\# If you set your pager option like below, you would be able to check the results in your default editor which is "notepad."}
\DocumentationTok{\#\# This option can be instrumental in other functions using pager option like \textasciigrave{}reticulate::py\_help()\textasciigrave{}.}
\FunctionTok{options}\NormalTok{(}\AttributeTok{pager =} \FunctionTok{getOption}\NormalTok{(}\StringTok{"editor"}\NormalTok{))}

\CommentTok{\# Useful option 3}
\DocumentationTok{\#\# Playing beep sound per error and use \{rlang\} style error trace:}
\FunctionTok{options}\NormalTok{(}\AttributeTok{error =} \ControlFlowTok{function}\NormalTok{() \{}
\NormalTok{  rlang}\SpecialCharTok{::}\FunctionTok{entrace}\NormalTok{()}
\NormalTok{  beepr}\SpecialCharTok{::}\FunctionTok{beep}\NormalTok{()}
\NormalTok{\})}

\CommentTok{\# Useful option 4}
\DocumentationTok{\#\# Turn off unicode for better screen reader readability}
\FunctionTok{options}\NormalTok{(}\AttributeTok{cli.unicode =} \ConstantTok{FALSE}\NormalTok{)}

\CommentTok{\# Useful option 5}
\DocumentationTok{\#\# Always use RStudio CRAN mirror for downloading packages}
\FunctionTok{options}\NormalTok{(}\AttributeTok{repos =} \FunctionTok{c}\NormalTok{(}\AttributeTok{CRAN =} \StringTok{"https://cran.rstudio.com"}\NormalTok{))}

\CommentTok{\# Useful option 6}
\DocumentationTok{\#\# Give more information about the LaTeX error}
\FunctionTok{options}\NormalTok{(}\AttributeTok{tinytex.verbose =} \ConstantTok{TRUE}\NormalTok{)}

\CommentTok{\# Specific options for VSCode:}

\DocumentationTok{\#\# Use VSCode as the default R editor}
\FunctionTok{options}\NormalTok{(}\AttributeTok{editor =} \StringTok{"code"}\NormalTok{)}

\DocumentationTok{\#\# VSCode R initializer and other options}
\ControlFlowTok{if}\NormalTok{ (}\FunctionTok{interactive}\NormalTok{() }\SpecialCharTok{\&\&} \FunctionTok{Sys.getenv}\NormalTok{(}\StringTok{"RSTUDIO"}\NormalTok{) }\SpecialCharTok{==} \StringTok{""}\NormalTok{) \{}
  \CommentTok{\# Load \textasciigrave{}init.R\textasciigrave{} provided by vscode{-}r extension: This is required to attach any user{-}custom R terminal}
  \FunctionTok{source}\NormalTok{(}\FunctionTok{file.path}\NormalTok{(}\FunctionTok{Sys.getenv}\NormalTok{(}\ControlFlowTok{if}\NormalTok{ (.Platform}\SpecialCharTok{$}\NormalTok{OS.type }\SpecialCharTok{==} \StringTok{"windows"}\NormalTok{) }\StringTok{"USERPROFILE"} \ControlFlowTok{else} \StringTok{"HOME"}\NormalTok{), }\StringTok{".vscode{-}R"}\NormalTok{, }\StringTok{"init.R"}\NormalTok{))}

  \CommentTok{\# Use an external browser for displaying html files, such as \{gt\} and \{xaringan\}}
  \FunctionTok{options}\NormalTok{(}\AttributeTok{vsc.viewer =} \ConstantTok{FALSE}\NormalTok{)}

  \CommentTok{\# Use an external browser for web apps, such as \{shiny\}}
  \FunctionTok{options}\NormalTok{(}\AttributeTok{vsc.browser =} \ConstantTok{FALSE}\NormalTok{)}

  \CommentTok{\# View help page in an external browser}
  \FunctionTok{options}\NormalTok{(}\AttributeTok{vsc.helpPanel =} \ConstantTok{FALSE}\NormalTok{)}

  \CommentTok{\# Use original data viewer (i.e., \textasciigrave{}utils::View()\textasciigrave{})}
  \FunctionTok{options}\NormalTok{(}\AttributeTok{vsc.view =} \ConstantTok{FALSE}\NormalTok{)}
\NormalTok{\}}

\DocumentationTok{\#\# Define code style according to \{tidyverse\}}
\FunctionTok{options}\NormalTok{(}\AttributeTok{languageserver.formatting\_style =} \ControlFlowTok{function}\NormalTok{(options) \{}
\NormalTok{  styler}\SpecialCharTok{::}\FunctionTok{tidyverse\_style}\NormalTok{(}\AttributeTok{indent\_by =} \DecValTok{2}\NormalTok{L)}
\NormalTok{\})}

\DocumentationTok{\#\# Define a more accessible view function}
\NormalTok{View }\OtherTok{\textless{}{-}} \ControlFlowTok{function}\NormalTok{(...) \{}
\NormalTok{  DT}\SpecialCharTok{::}\FunctionTok{datatable}\NormalTok{(...)}
\NormalTok{\}}
\end{Highlighting}
\end{Shaded}

\begin{enumerate}
\def\labelenumi{\arabic{enumi}.}
\setcounter{enumi}{6}
\item
  Press \texttt{Ctrl+S} (on Windows) or \texttt{Command+S} (on Mac) to
  save the file.
\item
  Press \texttt{Ctrl+W} (on Windows) or \texttt{Command+W} (on Mac) to
  close the file.
\item
  Close R terminal by pressing \texttt{Ctrl+W} (on Windows) or
  \texttt{Command+W} (on Mac). You may be prompted to confirm whether
  you want to close the terminal. If so, press \texttt{Enter} key to
  confirm.
\end{enumerate}

\hypertarget{sec-confirm-installation}{%
\section{Verifying R and VSCode
Installation}\label{sec-confirm-installation}}

To verify whether R and VSCode are installed correctly, you can follow
the steps below:

\begin{enumerate}
\def\labelenumi{\arabic{enumi}.}
\item
  Open VSCode. You can do this by pressing \texttt{Windows+R} and type
  ``code'' without the quotes and press enter. On Mac, you can press
  \texttt{Command+Space} and type ``code'' without the quotes and press
  enter.
\item
  In VSCode, create an untitled file by pressing \texttt{Ctrl+N} (on
  Windows) or \texttt{Command+N} (on Mac).
\item
  In the opened untitled file, type the following code:
\end{enumerate}

\begin{Shaded}
\begin{Highlighting}[]
\FunctionTok{print}\NormalTok{(}\StringTok{"Hello world!"}\NormalTok{)}
\end{Highlighting}
\end{Shaded}

\begin{enumerate}
\def\labelenumi{\arabic{enumi}.}
\setcounter{enumi}{3}
\item
  Save this untitled file as ``test.R'' without the quotes. You can save
  the file by pressing \texttt{Ctrl+S} (on Windows) or
  \texttt{Command+S} (on Mac). You will be prompted to enter the file
  name. If you would like to save the file in a specific path, you can
  include it in the file name. For example, if you would like to save
  the file in your C drive, you can type ``C:\test.R'' without the
  quotes and hit \texttt{Enter}.
\item
  In order to execute \texttt{print("Hello\ world!")} code, place your
  cursor on the line where the code is located via \texttt{UpArrow} or
  \texttt{DownArrow} keys.
\item
  You can execute the \texttt{print("Hello\ world!")} code by pressing
  \texttt{Ctrl+Enter} (on Windows) or \texttt{Command+Enter} (on Mac).
  The code will be sent to the R terminal and the result will be
  displayed in the R terminal.
\item
  Press \texttt{Ctrl+Tab} to switch to the R terminal. You will hear
  something like ``R Interactive Terminal.'' Your initial focus will be
  on the R terminal input area where you can type R code. If you press
  \texttt{Shift+Tab} key, the focus moves to the terminal output area
  where you can read the result of the executed code via standard arrow
  navigation. If you press \texttt{Tab} key, the focus moves back to the
  R terminal input area. I recommend you keep the focus on the R
  terminal output area.
\item
  From R terminal, press \texttt{Ctrl+Tab} to come back to the test.R
  file. You will hear something like ``test dot R.'' Now you can switch
  between the R terminal and the test.R file back and forth by pressing
  \texttt{Ctrl+Tab} key.
\item
  In the test.R file, type the following code and execute the code by
  pressing \texttt{Ctrl+Enter} (on Windows) or \texttt{Command+Enter}
  (on Mac).
\end{enumerate}

\begin{Shaded}
\begin{Highlighting}[]
\FunctionTok{stop}\NormalTok{()}
\end{Highlighting}
\end{Shaded}

\begin{enumerate}
\def\labelenumi{\arabic{enumi}.}
\setcounter{enumi}{9}
\item
  You will hear chiming sound that indicates an error. If you press
  \texttt{Ctrl+Tab} key, you will hear ``R Interactive Terminal.'' If
  your last focus was on the R terminal output area, you can navigate
  the executed code via standard arrow navigation. We intentionally
  stopped the code execution to test whether the error beep sound is
  played correctly.
\item
  Once you have verified all of these steps, you can close R terminal by
  pressing \texttt{Ctrl+W} (on Windows) or \texttt{Command+W} (on Mac).
  You can also close the test.R file by pressing \texttt{Ctrl+W} (on
  Windows) or \texttt{Command+W} (on Mac). Close VSCode by pressing
  \texttt{Alt+F4} key on Windows or \texttt{Command+Q} key on Mac.
\end{enumerate}



\end{document}
